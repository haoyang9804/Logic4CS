\subsection{Proof System}


\begin{definition}[consistent]
  A set of \wff $\Sigma$ is consistent if $\Sigma\vdash\alpha$ and $\Sigma\vdash\neg\alpha$ does not hold together.
\end{definition}

\begin{mdframed}
Another definition of consistency is that $\Sigma\nvdash\perp$ where $\perp$ is a contradiction\footnote{https://planetmath.org/consistent}.
\end{mdframed}

\begin{claim}
  A set of \wff $\Sigma$ is consistent iff $\exists\alpha, \Sigma\nvdash\alpha$.
\end{claim}

\begin{proof}

  $\rightarrow$: pick any $\alpha$, either $\Sigma\nvdash\alpha$ or $\Sigma\vdash\alpha$, then there exists one that is not proved by $\Sigma$.

$\leftarrow$: assume that for some $beta$, $\Sigma\nvdash\beta$, if the first definition is violated, then, for some $\alpha$, $\Sigma\vdash\alpha$ and $\Sigma\vdash\neg\alpha$, then $\Sigma\vdash\Sigma\cup\{\alpha,\neg\alpha\}\vdash\{\alpha,\neg\alpha\}$. $\{\alpha, \neg\alpha\}\vdash\beta$ for every $\beta$, contradicting the assumption.
\end{proof}

\begin{corollary}
If $\Sigma\subseteq\Sigma'$, then if $\Sigma$ is consistent, then so does $\Sigma'$.
\end{corollary}

\begin{definition}[Soundness]
If $\vdash\alpha$, then $\alpha$ is a tautology.
\begin{mdframed}
In another word, every theorem is a tautology.
\end{mdframed}
\end{definition}

\begin{corollary}
$\nvdash P$, since $P$ is a propositional variable, which is not a tautology.
\end{corollary}

The universe $U$ (the set which includes everything) is the least consistent, since it contains everything. Therefore, for every $\alpha\in U$, $\neg\alpha\in U$, $U$ can prove both. It has the most unsteady stance:).

$\vdash$ is about syntax while $\vDash$ is about semantics. $\vdash$ shows that though the machine doesn't know anything, it can reach some result by axioms and modus ponens. $\vDash$ shows that the result is true in the real world given by human being's assignment of truth values.

\begin{theorem}
In a sound proof system, every satisfiable $\Sigma$ is consistent. 
\end{theorem}

\begin{proof}
\bwoc, assume the satisfiable $\Sigma$ is inconsistent. Then, for some $\alpha$, both $\Sigma\vdash\alpha$ and $\Sigma\vdash\neg\alpha$. 
If $\Sigma$ is satisfiable, then for some truth assignment $V$, $V$ satisfies all \wff in $\Sigma$. By soundness, $\Sigma\vDash\alpha$ and $\Sigma\vDash\neg\alpha$. So for that assignment $V$, we get $V(\alpha)=T$ and $V(\neg\alpha)=T$, violating the truth table of negation.
\end{proof}

The above theorem builds a bridge from semantic to syntax.

\begin{theorem}[Extended Soundness]
If $Sigma\vdash\alpha$, then $\Sigma\vDash\alpha$.
\end{theorem}

\begin{corollary}
If any set of \wff is consistent, then, in particular $\emptyset$.
\end{corollary}


\begin{definition}[Maximally Consistent]
We say that $\Sigma$ is maximally consistent if $\Sigma$ is consistent, but, for every $\alpha$, either $\Sigma\vdash\alpha$ or $\Sigma\cup\{\alpha\}$ is inconsistent.
\end{definition}

\begin{mdframed}
Consistency tells us that if $\Sigma\vdash\alpha$, then $\Sigma\nvdash\alpha$ doesn't hold. But if $\Sigma\nvdash\alpha$, then we can't say $\Sigma\vdash\alpha$ holds. Consistency only ensures no contradiction.

Maximally consistency tells us that if $\Sigma\nvdash\alpha$, then $\Sigma\vdash\alpha$ must hold. The maximality nature ensures that the negation of every nonprovable \wff is provable.
\end{mdframed}

\begin{mdframed}
\begin{example}
Let $\Sigma\equiv\{P_1\}$ over the variables $P_1, P_2, \ldots$.
\begin{claim}
$\{P_1\}$ is consistent since it's satisfiable.
\end{claim}
\begin{claim}
$\{P_1\}$ is not maximally consistent
\end{claim}
\begin{proof}
  It suffices to show that $\{P_1\}\nvdash P_3$ and $\{P_1, P_3\}$ is consistent.

  By soundness, it suffices to show that $\{P_1\}\vDash P_3$. Just find some truth assignment that $V$ \st $V(P_1)=T$ and $V(P_3)=F$. So $\{P_1\}\nvdash P_3$.
  
  Since $\{P_1, P_3\}$ is satisfiable, so it's consistent.
\end{proof}
\end{example}
\end{mdframed}

\begin{lemma}
For every consistent $\Sigma$, there exists a maximally consistent $\Sigma' \supseteq \Sigma$.
\end{lemma}

\begin{proof}
Let $\{\alpha_1, \ldots, \alpha_i, \alpha_{i+1}, \ldots\}$ is a set of \wff over $\{P_1,P_2,\ldots,P_i,\ldots\}$
\end{proof}

\begin{definition}[Monotonicity]
  $\forall \Sigma, \Sigma', \alpha$, if $\Sigma\vdash\alpha, \Sigma\subseteq\Sigma'$, then $\Sigma'\vdash\alpha$.
\end{definition}

\begin{theorem}[Deduction Theorem]
  $\Sigma \cup \{ \alpha \} \vdash \beta$ iff $\Sigma \vdash \alpha \rightarrow \beta$.
\end{theorem}


\begin{theorem}
  Any consistent set of \wff $\Sigma$ can be extended to $\Sigma$' \st $\Sigma' \supseteq \Sigma$ that is maximally consistent.
\end{theorem}


\begin{proof}
We construct a sequence of set of \wff. $\Sigma_0 \subseteq \Sigma_1 \subseteq \Sigma_2 \subseteq \ldots \subseteq \Sigma_i \subseteq \Sigma_{i+1}$ \st 1) $\Sigma_0 = \Sigma$ and 2) For all $\Sigma_i$, $\Sigma_i$ is consistent, and 3) for fixed enumeration of all \wff., $\alpha_1, \alpha_2, \ldots, \alpha_n$, for all $i$, either $\Sigma_i \vdash \alpha$ or $\Sigma_i \vdash \neg \alpha$.

Assume $\Sigma_i$ is defined and meets the requirements, let $\Sigma_{i+1}$ be $\Sigma_i$ if $\Sigma_i \vdash \neg \alpha_{i}$, otherwise $\Sigma_{i+1} = \Sigma_i \cup \{ \alpha_i \}$. 

Now we want to prove if $\Sigma_i$ meets the requirement, then so will $\Sigma_{i+1}$.
Requirement 1) is trivial. Requirement 2) and 3) are simultaneously proved by showing $\Sigma_{i+1} \nvdash \neg \alpha_i$ if $\Sigma_{i+1} = \Sigma_i \cup \{\alpha_i\}$, which is shown by the following claim.

\begin{claim}
  $\forall \Sigma, \alpha$, if $\Sigma \cup \{ \alpha \} \vdash \neg \alpha$, then $\Sigma \vdash \neg \alpha$. 
\end{claim}
\begin{proof}
  To prove this claim, it suffices to show $\vdash (\alpha \rightarrow \neg \alpha) \rightarrow \neg \alpha$ based on deduction theorem, which is a tautology.
\end{proof}

Now we show the $\Sigma'=\cup_{i=1}^\infty \Sigma_i$ is maximally consistent.
To prove it, it suffices to show that $\forall \alpha_i, i \in N$, $\Sigma' \vdash \alpha$ or $\Sigma' \vdash \neg \alpha$.
By requirement 2), $\Sigma_i \vdash \alpha_i$ or $\Sigma_i \vdash \neg \alpha_i$. Now we are going to show $\Sigma' \vdash \alpha_i$ or $\Sigma' \vdash \neg \alpha_i$.

\bwoc, suppose $\Sigma'$ is inconsistent. In this case, for some $\alpha$, $\Sigma'\vdash\alpha\ and\ \Sigma'\vdash\neg\alpha$. Let $\beta_1,\ldots\beta_k$ be the proof of $\alpha$ from $\Sigma'$ and $\gamma_1,\ldots,\gamma_l$ be the proof of $\neg\alpha$ from $\Sigma'$.

Each $\beta_i$ that is an assumption from $\Sigma'$ belongs to some $\Sigma_{m_i}$, and each $\gamma_i$ that is an assumption from $\Sigma'$ belongs to some $\Sigma_{m_j}$.
Since both formal proofs of $\alpha$ and of $\neg\alpha$ are finite, there is some $i^*$ that is bigger than all of these $m_i$'s and $m_j$'s. Therefore, for each $beta_i$ or $gemma_j$ that are used as assumptions, $\beta_i, \gamma_j \in \Sigma_{i^*}$. Now by monotonicity, $\Sigma_{i^*} \vdash \alpha$ and $\Sigma_{i^*} \vdash \neg \alpha$, which is a contradiction to requirement 2).

\end{proof}

\begin{theorem}
  In a sound proof system, every consistent $\Sigma$ is satisfiable.
\begin{mdframed}
Also called the completeness theorem.
\end{mdframed}
\end{theorem}


\begin{proof}
Let $\Sigma'$ be a maximally consistent set of \wff.s \st $\Sigma\subseteq\Sigma'$. Define a truth assignment $V_{\Sigma'}$ as follows: $V_{\Sigma'}(P_i) = T$ iff $\Sigma' \vdash P_i$ where $P_i$ is a propositional variable.
Now extend it into  $V_{\Sigma'}(\alpha_i) = T$ iff $\Sigma' \vdash \alpha_i$.
\begin{claim}
For every formula $\alpha$, $\overline{V}_{\Sigma'}(\alpha)=T$ iff $\Sigma'\vdash\alpha$, where $\overline{V}_{\Sigma'}$ is the extension of $V_{\Sigma'}$ in the infinite set of \wff, and $\overline{V}_{\Sigma'}$ is the truth assignment that satisfies $\Sigma'$.
\end{claim}
\begin{proof}
By generalized induction on the set of all \wff:
$I(\mathcal{P}, \{\rightarrow, \neg\})$ where $\mathcal{P}$ is the set of propositional variables, and $\{\rightarrow, \neg\}$ are the adequate connectives.
$I(\mathcal{P}, \{\rightarrow, \neg\})$ means that all \wff s that are built from $\mathcal{P}$ and $\{\rightarrow, \neg\}$.

\textbf{Induction base}. $\alpha\in\mathcal{P}$.

$\leftarrow$: $\overline{V}_{\Sigma'}(\alpha)=T$, then $\Sigma'\vdash\alpha$ by definition of $V_{\Sigma'}$.

$\rightarrow$: Suppose $\Sigma'\nvdash\alpha$, then $\overline{V}_{\Sigma'}(\alpha)=F$ still by the definition.

\begin{mdframed}
Remember the relation between truth assignment and the proof symbol. $\Sigma\vdash\alpha$ means that for every truth assignment $V$ \st $V(\Sigma)=T$, then $V(\alpha)=T$, which can be simplified as $V_{\Sigma}(\alpha)=T$.
\end{mdframed}

\textbf{Induction step}. Assume the claim holds for $\alpha$ and for $\beta$, we need to show that for $\neg\alpha$ and $\alpha\rightarrow\beta$.

$\neg\alpha$:

$\leftarrow$:
If $\Sigma'\vdash\neg\alpha$, by the consistency, $\Sigma'\nvdash\alpha$. So by the induction hypothesis, $\overline{V}_{\Sigma'}(\alpha)=F$, so by the truth table, $\overline{V}_{\Sigma'}(\neg\alpha)=T$.

$\rightarrow$:
Assume $\Sigma'\nvdash\neg\alpha$, then by its maximality, $\Sigma'\vdash\alpha$. So by the induction hypothesis, $\overline{V}_{\Sigma'}(\alpha)=T$, so by the truth table, $\overline{V}_{\Sigma'}(\neg\alpha)=F$.

$\alpha\rightarrow\beta$:

$\leftarrow$: $\Sigma'\nvdash(\alpha\rightarrow\beta)$,
Since $\vdash\beta\rightarrow(\alpha\rightarrow\beta)$, $\Sigma'\nvdash\beta$. 
By the induction hypothesis, $\overline{V}_{\Sigma'}(\beta)=F$.
Now we need to show that $\overline{V}_{\Sigma'}(\alpha)=T$. 
\bwoc, assume $\overline{V}_{\Sigma'}(\alpha)=F$.
By the induction hypothesis, $\Sigma\nvdash\alpha$, and by the maximality of $\Sigma'$, $\Sigma'\vdash\neg\alpha$. And by $\vdash\neg\alpha\rightarrow(\alpha\rightarrow\beta)$, $\Sigma'\vdash(\alpha\rightarrow\beta)$, which contradicts $\Sigma'\nvdash(\alpha\rightarrow\beta)$. Therefore, $\overline{V}_{\Sigma'}(\alpha)=F$, so this case is proved.

$\rightarrow$: $\Sigma'\vdash(\alpha\rightarrow\beta)$.
Assume $\Sigma'\vdash\alpha$, in which case $\Sigma'\vdash\beta$ by modus ponens. Using the induction hypothesis, $\overline{V}_{\Sigma'}(\alpha)=T$ and $\overline{V}_{\Sigma'}(\beta)=T$, so $\overline{V}_{\Sigma'}(\alpha\rightarrow\beta)=T$.
Otherwise, $\Sigma'\nvdash\alpha$, so by the induction hypothesis, $\overline{V}_{\Sigma'}(\alpha)=F$, so by truth table, $\overline{V}_{\Sigma'}(\alpha\rightarrow\beta)=T$.
\end{proof}
\end{proof}


\begin{proof}
$\Sigma$ is satisfiable by all-truth assignemnt, so it's consistent.
To show it's maximally consistent, we need to show for every $\alpha$, if $\Sigma\nvdash\alpha$, then $\Sigma\cup\{\alpha\}$ is inconsistent.
If $\Sigma\nvdash\alpha$, then $\Sigma\nvDash\alpha$, then $\Sigma\cup\{\alpha\}$ is unsatisfiable, so it's inconsistent.

Based on the definition of the truth table, $V_{\Sigma}=T$.

\end{proof}